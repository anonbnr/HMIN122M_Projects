\documentclass[a4paper,12pt,usenames,dvipsnames]{beamer}
\usepackage[utf8]{inputenc}
\usepackage[T1]{fontenc}
\usepackage[french]{babel}
\usepackage{xcolor}
\usepackage{pifont}
\usetheme{Singapore} %Boadilla | Bergen | Madrid | Antibes | Hannover | Singapore | Warsaw

\newcommand{\cmark}{\ding{51}}
\newcommand{\xmark}{\ding{55}}
%----------------------------------------------------------------------------------------
%   TITLE INFORMATION
%----------------------------------------------------------------------------------------
\title{Entrepôts de données pour \textit{Tam voyages}}
\subtitle{HMIN122M -- Entrepôts de Données et Big-Data}
\author{B. Rima \and J. Saba \and T. Shaqura \and J. Bourgin}
\institute[UM]{M1 Informatique AIGLE}
\date{\today}
\logo{\includegraphics[scale=0.2]{images/umLogo.png}}

%----------------------------------------------------------------------------------------
%   OUTLINE
%----------------------------------------------------------------------------------------
\AtBeginSection[]{
  \begin{frame}{Sommaire}
  \tableofcontents[currentsection]
  \end{frame}
}

\begin{document}
%----------------------------------------------------------------------------------------
%   TITLE FRAME
%----------------------------------------------------------------------------------------
\begin{frame}
\titlepage
\end{frame}
%----------------------------------------------------------------------------------------
%   INTRODUCTION
%----------------------------------------------------------------------------------------
\section{Introduction}
\begin{frame}{Objectifs de \textit{tam-voyages}}{Introduction}
\begin{itemize}
  \item<1-> augmenter le taux de vente des tickets;
  \item<2-> augmenter le taux d'abonnements;
  \item<3-> améliorer la qualité de service;
  \item<4-> réduire les dépenses;
  \item<5-> $\dots$
\end{itemize}
\end{frame}

\begin{frame}{Problématiques}{Introduction}
  \begin{block}{Problématique 1}
    \og \textit{Comment peut-on tirer partie de la fréquentation des véhicules en se basant sur la circulation du réseau afin d'améliorer la qualité de service ?} \fg
  \end{block}

  \begin{block}{Problématique 2}
    \og \textit{Comment peut-on suivre l'évolution et la maintenance des matériaux de manière à réduire les dépenses associées ?} \fg
  \end{block}
\end{frame}

\begin{frame}{Actions et opérations possibles par \textit{tam-voyages}}{Introduction}
  \begin{itemize}
    \item<1-> \textcolor{OliveGreen}{voyages};
    \item<2-> \textcolor{OliveGreen}{maintenance de véhicules};
    \item<3-> vente de tickets et abonnements;
    \item<4-> amendes;
    \item<5-> $\dots$
  \end{itemize}
\end{frame}

\begin{frame}{Actions et opérations considérées}{Introduction}
  \begin{block}{Voyages}
    \texttt{Le voyage d'un voyageur via un véhicule d'une ligne du réseau à une heure et une date donnée}.
  \end{block}

  \begin{examples}
    \begin{itemize}
      \item<1-> \textit{le nombre de voyages via ticket par chaque bus pour le mois de juillet}
      \item<2-> \textit{l'arrêt le plus fréquenté par chaque ligne du réseau}
    \end{itemize}
  \end{examples}
\end{frame}

\begin{frame}{Actions et opérations considérées}{Introduction}
  \begin{block}{Maintenance de véhicules}
    \texttt{Chaque transaction effectuée lors de la maintenance d'un véhicule à une heure et une date donnée}.
  \end{block}

  \begin{examples}
    \begin{itemize}
      \item<1-> \textit{le coût total de maintenance de chaque véhicule}
      \item<2-> \textit{le nombre total de maintenances effectuées par véhicule pour les 6 dernier mois.}
    \end{itemize}
  \end{examples}
\end{frame}

\section{Modélisation}
\begin{frame}{Contexte du projet}{Introduction}
\end{frame}

\section{Implémentation}
\begin{frame}{Contexte du projet}{Introduction}
\end{frame}

\section{Conclusion}
\begin{frame}{Contexte du projet}{Introduction}
\end{frame}
\end{document}
