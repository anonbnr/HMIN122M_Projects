\documentclass[usenames,dvipsnames]{beamer}
\usepackage[utf8]{inputenc}
\usepackage[T1]{fontenc}
\usepackage[french]{babel}
\usepackage{xcolor}
\usepackage{pifont}
\usetheme{Singapore} %Boadilla | Bergen | Madrid | Antibes | Hannover | Singapore | Warsaw

\newcommand{\cmark}{\ding{51}}
\newcommand{\xmark}{\ding{55}}
%----------------------------------------------------------------------------------------
%   TITLE INFORMATION
%----------------------------------------------------------------------------------------
\title{Entrepôts de données pour \textit{Tam voyages}}
\subtitle{HMIN122M -- Entrepôts de Données et Big-Data}
\author{B. Rima \and J. Saba \and T. Shaqura \and J. Bourgin}
\institute[UM]{M1 Informatique AIGLE}
\date{\today}

\begin{document}
%----------------------------------------------------------------------------------------
%   TITLE FRAME
%----------------------------------------------------------------------------------------
\begin{frame}
\titlepage
\end{frame}
%----------------------------------------------------------------------------------------
%   OUTLINE
%----------------------------------------------------------------------------------------
\begin{frame}{Sommaire}
\tableofcontents
\end{frame}
%----------------------------------------------------------------------------------------
%   INTRODUCTION
%----------------------------------------------------------------------------------------
\section{Introduction}
\begin{frame}{Contexte du projet}{Introduction}
\end{frame}

\section{Modélisation}
\begin{frame}{Contexte du projet}{Introduction}
\end{frame}

\section{Implémentation}
\begin{frame}{Contexte du projet}{Introduction}
\end{frame}

\section{Conclusion}
\begin{frame}{Conclusion (1/3)}
Comme nous avons vu par la présentation que notre model répondu aux problématiques. 
\begin{block}{Perspectives}
Bien que notre model nous a permis de réaliser des analyses principal nous avons constaté qu’il était assez limité pour les 2 raisons :
\begin{itemize}
    \item Incapable de calculer le montant exact du chiffre d’affaires de tam-voyages. 
    \item Difficile à connaître la fréquentation de chaque trajet effectué par un véhicule.
\end{itemize}
\end{block}
\newpage

\end{frame}
\begin{frame}{Conclusion (2/3)}
\begin{block}{Donc nous proposon la perspective d'évolution suivante: }
\begin{itemize}
    \item Data Marts  pour la vente des tickets et les abonnements
    \begin{itemize}
    \item 	L’ajout d’un Data mart ayant comme action la vente d’un ticket à 
            \begin{itemize}
                \item un voyageur
                \item une date donnée
                \item une mesure désignant le prix du ticket vendu
            \end{itemize}
    \item 	L’ajout d’un Data mart ayant comme action l’abonnement à 
             \begin{itemize}
            \item un voyageur
            \item une date donnée 
            \item mesures désignant les frais de l’abonnement et sa durée de validité
            \end{itemize}
\end{itemize}
\end{itemize}
\end{block}
\end{frame}



\begin{frame}{Conclusion (3/3)}
\begin{itemize}
    \item Data Mart pour les trajets effectués par des véhicules
        \begin{itemize}
        \item 	l’ajout d’un Data Mart tel que chaque fait soit désigné par :
                \begin{itemize}
                \item un trajet effectué 
                \item un véhicule sur une ligne du réseau de transport 
                \item une date et une heure donnée
                \item sans mesure
                \end{itemize}
        \item	l’idée à modifier la table des voyages pour raffiner les analyses de manière à inclure des informations supplémentaires sur les trajets
        \item 	Un fait dans ladite table désignera ainsi un voyage effectué par un voyageur dans un véhicule faisant un trajet sur une ligne du réseau de transport à une date et une heure donnée
        \item 	il faut lier la table des voyages avec la table des trajets 
        \end{itemize}
\end{itemize}
\end{frame}

\end{document}
