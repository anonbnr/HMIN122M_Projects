\documentclass[a4paper,12pt]{article}
\usepackage[utf8]{inputenc}
\usepackage[francais]{babel}
\usepackage[T1]{fontenc}
\usepackage{graphicx}
\usepackage[colorlinks,urlcolor=blue]{hyperref} %hyperlinks
\usepackage[nottoc,notlot,notlof]{tocbibind} %To bind the table of contents to the bibligoraphy
\usepackage[page,toc,titletoc,title]{appendix} %To add appendices to the document
\usepackage{packages/tikz-uml} %UML elements

\title{
  HMIN122M Rendu \\
  \large TP1
}
\author{Bachar Rima \and Joseph Saba}
\date{\today}

\begin{document}
\pagestyle{plain}

\maketitle

{
  \hypersetup{linkcolor=black}
  \tableofcontents
}

\section{Modèle conceptuel : UML des cahiers des charges}
\subsection{Cahier des charges \#1 : les photos}
\begin{figure}[!ht]
  \centering
  \resizebox{\textwidth}{!}{
    \begin{tikzpicture}
      \umlclass[x=-8, y=4]{AppareilPhoto}{
        \underline{model}: \textsc{varchar}\\
        type: \textsc{varchar}\\
        resolution\_x: \textsc{number}\\
        lens: \textsc{varchar}
      }{}
      \umlclass[x=0, y=-2]{Photo}{
        \underline{code}: \textsc{integer}\\
        lieu: \textsc{sdo\_geometry}\\
        date: \textsc{date}\\
        chemin: \textsc{varchar}
      }{}
      \umlclass[x=0, y=4]{Licence}{
        \underline{id}: \textsc{Integer} \{auto\}\\
        titre: \textsc{varchar}\\
        droits: \textsc{varchar}
      }{}
      \umlclass[x=0, y=-8]{Configuration}{
        \underline{id}: \textsc{Integer} \{auto\}\\
        ouverture\_focale: \textsc{number}\\
        temps\_exposition: \textsc{varchar}\\
        flash: \textsc{boolean}\\
        distance\_focale: \textsc{number}
      }{}
      \umlclass[x=8, y=2]{Utilisateur}{
        \underline{id}: \textsc{varchar} \{hash\}\\
        nom: \textsc{varchar}\\
        prenom: \textsc{varchar}\\
        username: \textsc{varchar}\\
        email: \textsc{varchar}\\
        date\_naissance: \textsc{date}\\
        date\_inscription: \textsc{date}\\
        pays: \textsc{varchar}
      }{}
      \umlassoc[mult1=1, pos1=0.1, mult2=0..N, pos2=2.7, geometry =-|-, stereo=prend, pos stereo=1.2]{Photo}{AppareilPhoto}
      \umluniassoc[mult1=0..N, mult2=1, stereo=possede\_brevet]{Photo}{Licence}
      \umlassoc[name=assoc, mult1=1, pos1=0.1, mult2=0..N, pos2=2.7, geometry=-|-, stereo=publie, pos stereo=1.2]{Utilisateur}{Photo}
      \umlassocclass[x=8, y=-4, geometry=-|]{Publie}{assoc-1}{
        date\_publication: \textsc{date}\\
      }{}
      \umlunicompo{Photo}{Configuration}
    \end{tikzpicture}
  }
  \caption{Cahier des charges \#1 : les photos}
\end{figure}

\newpage

\subsection{Cahier des charges \#2 : publications, albums, et galeries}
\begin{figure}[!ht]
  \centering
  \begin{tikzpicture}
    \umlclass[x=0, y=4, type=abstract]{PhotoCollection}{
      \underline{id}: \textsc{integer} \{auto\}\\
      date\_creation: \textsc{date}\\
      nom: \textsc{varchar}
    }{}
    \umlemptyclass[x=-2, y=0]{Album}
    \umlemptyclass[x=2, y=0]{Gallery}
    \umlemptyclass[x=0, y=-3]{Photo}
    \umlemptyclass[x=2, y=-6]{Utilisateur}
    \umlVHVinherit[name=inherit]{Album}{PhotoCollection}
    \umlVHVinherit{Gallery}{PhotoCollection}
    \umlnote[x=-5, y=2]{inherit-1}{contraintes: complet et disjoint}
    \umlNarynode[name=rangeG, x=3, y=-3, right]{range\_gallery}
    \umlassoc[mult2=0..N, pos2=2, geometry=-|]{Gallery}{rangeG}
    \umlassoc[mult2=0..N]{Photo}{rangeG}
    \umlassoc[mult2=0..N, pos2=2, geometry=-|]{Utilisateur}{rangeG}
    \umlNarynode[name=rangeA, x=-3, y=-3, left]{range\_album}
    \umlassoc[mult2=0..N, pos2=2, geometry=-|]{Album}{rangeA}
    \umlassoc[mult2=0..N]{Photo}{rangeA}
    \umlassoc[mult2=1, pos2=2, geometry=-|]{Utilisateur}{rangeA}
    \umlnote[x=-6, y=-6]{rangeA}{contraintes: id de Utilisateur = id du Proprietaire de l'Album}
  \end{tikzpicture}
  \caption{Cahier des charges \#2 : publications, albums, et galeries}
\end{figure}

\newpage

\subsection{Cahier des charges \#3 : les interactions entre utilisateurs}
\begin{figure}[!ht]
  \centering
  \resizebox{\textwidth}{!}{
    \begin{tikzpicture}
      \umlclass[x=0, y=4, type=abstract]{ContenuNumerique}{
        \underline{id}: \textsc{integer} \{auto\}
      }{}
      \umlclass[x=8, y=0]{Comment}{
        date: \textsc{date}\\
        content: \textsc{text}
      }{}
      \umlclass[x=-8, y=0]{Photo}{
        \underline{code}: \textsc{integer}\\
        lieu: \textsc{sdo\_geometry}\\
        date: \textsc{date}\\
        chemin: \textsc{varchar}\\
        nb\_views: \textsc{integer}\\
      }{}
      \umlemptyclass[x=0, y=0]{Discussion}
      \umlemptyclass[x=0, y=-6]{Utilisateur}
      \umlemptyclass[x=-8, y=-6]{Tag}
      \umlVHVinherit{Photo}{ContenuNumerique}
      \umlVHVinherit{Comment}{ContenuNumerique}
      \umlunicompo{Discussion}{Comment}
      \umlassoc[mult1=0..N, mult2=1..N, stereo=etiquette]{Tag}{Photo}
      \umlassoc[mult1=1, mult2=0..1, stereo=annotate]{Photo}{Discussion}
      \umlassoc[mult1=0..N, mult2=0..N, stereo=likes]{Utilisateur}{Photo}
      \umlassoc[mult1=1, pos1=0.1, mult2=0..N, pos2=1.95, geometry=-|, stereo=commente, pos stereo=1.2]{Utilisateur}{Comment}
    \end{tikzpicture}
  }
  \caption{Cahier des charges \#3 : les interactions entre utilisateurs}
\end{figure}

\newpage

\section{Modèle relationnel}

\begin{appendices}
\section*{SQL}
\subsection{Requêtes de création de la base de données}

\subsection{Requêtes de population de la base de données}

\subsection{Requêtes de l'interrogation de la base de données }

\end{appendices}

\end{document}
