\documentclass[a4paper,12pt]{article}
\usepackage[utf8]{inputenc}
\usepackage[francais]{babel}
\usepackage[T1]{fontenc}
\usepackage{graphicx}
\usepackage{listingsutf8}
\usepackage[colorlinks,urlcolor=blue]{hyperref} %hyperlinks
\usepackage[nottoc,notlot,notlof]{tocbibind} %To bind the table of contents to the bibligoraphy
\usepackage[page,toc,titletoc,title]{appendix} %To add appendices to the document
\usepackage{packages/tikz-uml} %UML elements

\title{
  HMIN122M Rendu \\
  \large TP1
}
\author{Bachar Rima \and Joseph Saba}
\date{\today}

\lstset{
  language=SQL,
  basicstyle=\ttfamily\small,
  keywordstyle=\color{blue!60},
  commentstyle=\color{red!80}\upshape,
  stringstyle=\color{purple},
  showstringspaces=false,
  rulecolor=\color{black!30},
  numberstyle=\footnotesize,
  numbers=left,
  breaklines=true,
  breakatwhitespace=true,
  tabsize=3,
  frame=single,
  framextopmargin=2pt,
  framexbottommargin=2pt,
  backgroundcolor=\color{gray!10},
  captionpos=b,
  inputencoding=utf8,
  extendedchars=true,
  literate={á}{{\'a}}1 {é}{{\'e}}1 {í}{{\'i}}1 {ó}{{\'o}}1 {ú}{{\'u}}1 {Á}{{\'A}}1 {É}{{\'E}}1 {Í}{{\'I}}1 {Ó}{{\'O}}1 {Ú}{{\'U}}1 {à}{{\`a}}1 {è}{{\`e}}1 {ì}{{\`i}}1 {ò}{{\`o}}1 {ù}{{\`u}}1 {À}{{\`A}}1 {È}{{\'E}}1 {Ì}{{\`I}}1 {Ò}{{\`O}}1 {Ù}{{\`U}}1 {ä}{{\"a}}1 {ë}{{\"e}}1 {ï}{{\"i}}1 {ö}{{\"o}}1 {ü}{{\"u}}1 {Ä}{{\"A}}1 {Ë}{{\"E}}1 {Ï}{{\"I}}1 {Ö}{{\"O}}1 {Ü}{{\"U}}1 {â}{{\^a}}1 {ê}{{\^e}}1 {î}{{\^i}}1 {ô}{{\^o}}1 {û}{{\^u}}1 {Â}{{\^A}}1 {Ê}{{\^E}}1 {Î}{{\^I}}1 {Ô}{{\^O}}1 {Û}{{\^U}}1 {œ}{{\oe}}1 {Œ}{{\OE}}1 {æ}{{\ae}}1 {Æ}{{\AE}}1 {ß}{{\ss}}1 {ű}{{\H{u}}}1 {Ű}{{\H{U}}}1 {ő}{{\H{o}}}1 {Ő}{{\H{O}}}1 {ç}{{\c c}}1 {Ç}{{\c C}}1 {ø}{{\o}}1 {å}{{\r a}}1 {Å}{{\r A}}1 {€}{{\euro}}1 {£}{{\pounds}}1 {«}{{\guillemotleft}}1 {»}{{\guillemotright}}1 {ñ}{{\~n}}1 {Ñ}{{\~N}}1 {¿}{{?`}}1
}

\begin{document}
\pagestyle{plain}

\maketitle

{
  \hypersetup{linkcolor=black}
  \tableofcontents
}

\section{Modèle conceptuel : UML des cahiers des charges}
\subsection{Cahier des charges \#1 : les photos}
\begin{figure}[!ht]
  \centering
  \resizebox{\textwidth}{!}{
    \begin{tikzpicture}
      \umlclass[x=-8, y=4]{AppareilPhoto}{
        \underline{model}: \textsc{varchar}\\
        type: \textsc{varchar}\\
        resolution\_x: \textsc{number}\\
        lens: \textsc{varchar}
      }{}
      \umlclass[x=0, y=-2]{Photo}{
        \underline{id}: \textsc{integer}\\
        lieu: \textsc{sdo\_geometry}\\
        chemin: \textsc{varchar}
      }{}
      \umlclass[x=0, y=4]{Licence}{
        \underline{id}: \textsc{Integer} \{auto\}\\
        droits: \textsc{varchar}
      }{}
      \umlclass[x=0, y=-8]{Configuration}{
        \underline{id}: \textsc{Integer} \{auto\}\\
        ouverture\_focale: \textsc{number}\\
        temps\_exposition: \textsc{varchar}\\
        flash: \textsc{boolean}\\
        distance\_focale: \textsc{number}
      }{}
      \umlclass[x=8, y=2]{Utilisateur}{
        \underline{id}: \textsc{varchar} \{hash\}\\
        nom: \textsc{varchar}\\
        prenom: \textsc{varchar}\\
        username: \textsc{varchar}\\
        email: \textsc{varchar}\\
        date\_naissance: \textsc{date}\\
        date\_inscription: \textsc{date}\\
        pays: \textsc{varchar}
      }{}
      \umlassoc[mult1=1, pos1=0.1, mult2=0..N, pos2=2.7, geometry =-|-, stereo=prend, pos stereo=1.2]{Photo}{AppareilPhoto}
      \umluniassoc[mult1=0..N, mult2=1, stereo=possede\_brevet]{Photo}{Licence}
      \umlassoc[name=assoc, mult1=1, pos1=0.1, mult2=0..N, pos2=2.7, geometry=-|-, stereo=publie, pos stereo=1.2]{Utilisateur}{Photo}
      \umlassocclass[x=8, y=-4, geometry=-|]{Publie}{assoc-1}{
        date\_publication: \textsc{date}\\
      }{}
      \umlunicompo{Photo}{Configuration}
    \end{tikzpicture}
  }
  \caption{Cahier des charges \#1 : les photos}
\end{figure}

\newpage

\subsection{Cahier des charges \#2 : publications, albums, et galeries}
\begin{figure}[!ht]
  \centering
  \begin{tikzpicture}
    \umlclass[x=0, y=4, type=abstract]{PhotoCollection}{
      \underline{id}: \textsc{integer} \{auto\}\\
      nom: \textsc{varchar}
    }{}
    \umlemptyclass[x=-2, y=0]{Album}
    \umlemptyclass[x=2, y=0]{Gallery}
    \umlemptyclass[x=0, y=-3]{Photo}
    \umlemptyclass[x=8, y=-6]{Utilisateur}
    \umlVHVinherit[name=inherit]{Album}{PhotoCollection}
    \umlVHVinherit{Gallery}{PhotoCollection}
    \umlassoc[name=assoc, mult1=0..N, mult2=1, geometry=|-, stereo=cree]{Utilisateur}{PhotoCollection}
    \umlassocclass[x=5, y=2, geometry=-|]{Cree}{assoc-1}{
      date\_creation: \textsc{date}\\
    }{}
    \umlnote[x=-5, y=2]{inherit-1}{contraintes: complet et disjoint}
    \umlNarynode[name=rangeG, x=3, y=-3, right]{range\_gallery}
    \umlassoc[mult2=0..N, pos2=2, geometry=-|]{Gallery}{rangeG}
    \umlassoc[mult2=0..N]{Photo}{rangeG}
    \umlassoc[mult2=0..N, pos2=2, geometry=-|]{Utilisateur}{rangeG}
    \umlNarynode[name=rangeA, x=-3, y=-3, left]{range\_album}
    \umlassoc[mult2=0..N, pos2=2, geometry=-|]{Album}{rangeA}
    \umlassoc[mult2=0..N]{Photo}{rangeA}
    \umlassoc[mult2=1, pos2=2, geometry=-|]{Utilisateur}{rangeA}
    \umlnote[x=-6, y=-6]{rangeA}{contraintes: id de Utilisateur = id du Proprietaire de l'Album}
  \end{tikzpicture}
  \caption{Cahier des charges \#2 : publications, albums, et galeries}
\end{figure}

\newpage

\subsection{Cahier des charges \#3 : les interactions entre utilisateurs}
\begin{figure}[!ht]
  \centering
  \resizebox{\textwidth}{!}{
    \begin{tikzpicture}
      \umlclass[x=0, y=4, type=abstract]{ContenuNumerique}{
        \underline{id}: \textsc{integer} \{auto\}
      }{}
      \umlclass[x=8, y=0]{Commentaire}{
        \underline{id}: \textsc{integer} \{auto\} <<FK>>\\
        contenu: \textsc{text}
      }{}
      \umlclass[x=-8, y=0]{Photo}{
        \underline{id}: \textsc{integer} \{auto\} <<FK>>\\
        lieu: \textsc{sdo\_geometry}\\
        chemin: \textsc{varchar}\\
        nb\_views: \textsc{integer}\\
      }{}
      \umlclass[x=0, y=0]{Discussion}{
        \underline{id}: \textsc{integer} \{auto\}
      }{}
      \umlemptyclass[x=-10, y=6]{Utilisateur}
      \umlclass[x=-8, y=-6]{Tag}{
        \underline{id}: \textsc{integer} \{auto\}\\
        contenu: \textsc{varchar}
      }{}
      \umlVHVinherit{Photo}{ContenuNumerique}
      \umlVHVinherit{Commentaire}{ContenuNumerique}
      \umlunicompo{Discussion}{Commentaire}
      \umlassoc[name=assoc, mult1=1, pos1=0.1, mult2=0..N, pos2=1, geometry=-|-, stereo=publie]{Utilisateur}{ContenuNumerique}
      \umlassocclass[x=0, y=8, geometry=-|]{Publie}{assoc-1}{
        date\_publication: \textsc{date}\\
      }{}
      \umlassoc[mult1=0..N, mult2=1..N, stereo=marque]{Tag}{Photo}
      \umlassoc[mult1=1, mult2=0..1, stereo=annotate]{Photo}{Discussion}
      \umlassoc[geometry=|-, mult1=0..N, pos1=0.1, mult2=0..N, pos2=0.6, stereo=aime, pos stereo=0.4]{Utilisateur}{Photo}
      \umlassoc[mult1=0..N, pos1=0.1, angle1=90, mult2=0..N, pos2=0.7, angle2=140, loopsize=5cm, stereo=suit, pos stereo=0.5]{Utilisateur}{Utilisateur}
    \end{tikzpicture}
  }
  \caption{Cahier des charges \#3 : les interactions entre utilisateurs}
\end{figure}

\newpage

\section{Modèle relationnel}
\begin{itemize}
  \item \textbf{AppareilPhoto}(\underline{model}, type, resolution\_x, lens)
  \item \textbf{Licence}(\underline{id}, droits)
  \item \textbf{Configuration}(\underline{id}, ouverture\_focale, temps\_exposition, flash, distance\_focale)
  \item \textbf{Utilisateur}(\underline{id}, nom, prenom, username, email, date\_naissance, date\_inscription, pays)
  \item \textbf{ContenuNumerique}(\underline{id}, \textit{id\_utilisateur}, date\_publication)
  \item \textbf{PhotoCollection}(\underline{id}, nom, \textit{id\_utilisateur}, date\_creation)
  \item \textbf{Album}(\textit{\underline{id}})
  \item \textbf{Gallery}(\textit{\underline{id}})
  \item \textbf{Photo}(\textit{\underline{id}}, lieu, chemin, nb\_views, \textit{id\_appareil}, \textit{id\_licence}, \textit{id\_configuration})
  \item \textbf{Discussion}(\underline{id}, \textit{id\_photo})
  \item \textbf{Commentaire}(\textit{\underline{id}}, contenu, \textit{id\_discussion})
  \item \textbf{Tag}(\textit{\underline{id}}, contenu)
  \item \textbf{range\_gallery}(\textit{\underline{id\_gallery}}, \textit{\underline{id\_utilisateur}}, \textit{\underline{id\_photo}})
  \item \textbf{range\_album}(\textit{\underline{id\_album}}, \textit{\underline{id\_utilisateur}}, \textit{\underline{id\_photo}})\footnote{un \textit{trigger} est requis pour vérifier la contrainte d'intégrité désignant l'unicité du propriétaire des photos dans un album}
  \item \textbf{aime}(\textit{\underline{id\_utilisateur}}, \textit{\underline{id\_photo}})
  \item \textbf{suit}(\textit{\underline{id\_utilisateur$_1$}}, \textit{\underline{id\_utilisateur$_2$}})\footnote{Contrainte d'intégrité : \textit{id\_utilisateur$_1$} $\neq$ \textit{id\_utilisateur$_2$}}
  \item \textbf{marque}(\textit{\underline{id\_tag}}, \textit{\underline{id\_photo}})
\end{itemize}

\begin{appendices}
\section*{SQL}
\subsection{Requêtes de création des tables de la base de données}
\lstinputlisting{sql/tables.sql}
\lstinputlisting{sql/triggers.sql}

\subsection{Requêtes de population des tables de la base de données}
\lstinputlisting{sql/inserts.sql}

\subsection{Requêtes d'interrogation de la base de données}
\lstinputlisting{sql/queries.sql}

\end{appendices}

\end{document}
